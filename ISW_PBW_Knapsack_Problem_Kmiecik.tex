\documentclass[conference,compsoc]{IEEEtran}

% *** CITATION PACKAGES ***
%
\ifCLASSOPTIONcompsoc
  % IEEE Computer Society needs nocompress option
  % requires cite.sty v4.0 or later (November 2003)
  \usepackage[nocompress]{cite}
\else
  % normal IEEE
  \usepackage{cite}
\fi

% *** GRAPHICS RELATED PACKAGES ***
%
\ifCLASSINFOpdf
\else
\fi

% correct bad hyphenation here
\hyphenation{op-tical net-works semi-conduc-tor}


\begin{document}

\title{Optimizing the loading of a truck in order\\
to make best use of space\\
and increase revenue a single transport }


% author names and affiliations
% use a multiple column layout for up to three different
% affiliations
\author{
\IEEEauthorblockN{Lukasz Joksch}
\IEEEauthorblockA{Department of Electronics,\\Wroclaw University of Technology\\Wroclaw, Poland\\
E-mail:lukaszjok@gmail.com}
\and
\IEEEauthorblockN{Tomasz Kowalik}
\IEEEauthorblockA{Department of Electronics,\\Wroclaw University of Technology\\Wroclaw, Poland\\
tomasz\_kowalik@hotmail.com}
\and
\IEEEauthorblockN{Piotr Tazbir}
\IEEEauthorblockA{Department of Electronics,\\Wroclaw University of Technology\\Wroclaw, Poland\\
E-mail: - - - - - - - }
}

% conference papers do not typically use \thanks and this command
% is locked out in conference mode. If really needed, such as for
% the acknowledgment of grants, issue a \IEEEoverridecommandlockouts
% after \documentclass

% for over three affiliations, or if they all won't fit within the width
% of the page (and note that there is less available width in this regard for
% compsoc conferences compared to traditional conferences), use this
% alternative format:
% 
%\author{\IEEEauthorblockN{Michael Shell\IEEEauthorrefmark{1},
%Homer Simpson\IEEEauthorrefmark{2},
%James Kirk\IEEEauthorrefmark{3}, 
%Montgomery Scott\IEEEauthorrefmark{3} and
%Eldon Tyrell\IEEEauthorrefmark{4}}
%\IEEEauthorblockA{\IEEEauthorrefmark{1}School of Electrical and Computer Engineering\\
%Georgia Institute of Technology,
%Atlanta, Georgia 30332--0250\\ Email: see http://www.michaelshell.org/contact.html}
%\IEEEauthorblockA{\IEEEauthorrefmark{2}Twentieth Century Fox, Springfield, USA\\
%Email: homer@thesimpsons.com}
%\IEEEauthorblockA{\IEEEauthorrefmark{3}Starfleet Academy, San Francisco, California 96678-2391\\
%Telephone: (800) 555--1212, Fax: (888) 555--1212}
%\IEEEauthorblockA{\IEEEauthorrefmark{4}Tyrell Inc., 123 Replicant Street, Los Angeles, California 90210--4321}}




% use for special paper notices
%\IEEEspecialpapernotice{(Invited Paper)}




% make the title area
\maketitle

% As a general rule, do not put math, special symbols or citations
% in the abstract
\begin{abstract}
The abstract goes here.
\end{abstract}

% no keywords




% For peer review papers, you can put extra information on the cover
% page as needed:
% \ifCLASSOPTIONpeerreview
% \begin{center} \bfseries EDICS Category: 3-BBND \end{center}
% \fi
%
% For peerreview papers, this IEEEtran command inserts a page break and
% creates the second title. It will be ignored for other modes.
\IEEEpeerreviewmaketitle



\section{Introduction}
% no \IEEEPARstart
In many companies there is some problem with optimization of loading goods. This is very important multifaceted trouble, which concerns spending of money, waste of time and human resources management.
\subsection{Saving money}
When headmasters and managers do not implement optimization algorithms, work of their employees is ineffective. This in turn is associated with rising industry cost. If managers introduced new solution, they would use better company resources. % trzeci tryb warunkowy
The main advantage is that we are able to waste a place in trucks - developed software gives optimal arrangement of packages or select which loads pack into truck or not. Thanks to this solution is possible  to load up the lorry. Drivers drive less routes, so their chief spend less money on fuel and all maintenance costs of the car.
\subsection{Faster work results}
Another advantage is that the same work, which people were doing before implementing optimization algorithm, they do it in shorter time. This is due to the fact that they do not have to plan in real time position of next packages, because employees have generated list with optimal positions. What is more, optimization process is very efficient, because it is automated - once written application is used repeatedly.
\subsection{Responsible use of human resources}
The use of optimization work makes people more thoughtful, it reduces the effort in the process of loading trucks. People have more time to regeneration and rest between successive jobs.

\section{General overview of knapsack problem}
Knapsack problem, often also called rucksack problem, is the most popular question in optimization matter. The main objective is to pack a goods as much as is possible with maximal value and minimal weight. The knapsack problem is said to be a thief problem, because as he, we want to get the most valuable items and put it in limited space(knapsack).

\subsection{Definition}
Subsection text here. siemanko


\subsubsection{Subsubsection Heading Here}
Subsubsection text here.






\section{Conclusion}
The conclusion goes here.




% conference papers do not normally have an appendix



% use section* for acknowledgment
\ifCLASSOPTIONcompsoc
  % The Computer Society usually uses the plural form
  \section*{Acknowledgments}
\else
  % regular IEEE prefers the singular form
  \section*{Acknowledgment}
\fi


The authors would like to thank...





% trigger a \newpage just before the given reference
% number - used to balance the columns on the last page
% adjust value as needed - may need to be readjusted if
% the document is modified later
%\IEEEtriggeratref{8}
% The "triggered" command can be changed if desired:
%\IEEEtriggercmd{\enlargethispage{-5in}}

% references section

% can use a bibliography generated by BibTeX as a .bbl file
% BibTeX documentation can be easily obtained at:
% http://mirror.ctan.org/biblio/bibtex/contrib/doc/
% The IEEEtran BibTeX style support page is at:
% http://www.michaelshell.org/tex/ieeetran/bibtex/
%\bibliographystyle{IEEEtran}
% argument is your BibTeX string definitions and bibliography database(s)
%\bibliography{IEEEabrv,../bib/paper}
%
% <OR> manually copy in the resultant .bbl file
% set second argument of \begin to the number of references
% (used to reserve space for the reference number labels box)
\begin{thebibliography}{1}

\bibitem{IEEEhowto:kopka}
H.~Kopka and P.~W. Daly, \emph{A Guide to \LaTeX}, 3rd~ed.\hskip 1em plus
  0.5em minus 0.4em\relax Harlow, England: Addison-Wesley, 1999.

\end{thebibliography}




% that's all folks
\end{document}


